\documentclass{article}
\usepackage[utf8]{inputenc}

\newcommand{\tnum}{4}
\title{Tarea \tnum\\
      Algoritmos y Complejidad\\[2ex]
      }
\author{
        \textbf{Juan Parra Escobar}\\
        \textbf{201973094-0}
}

\date{17/03/2022}

% nº tarea: 4
% nombre: Juan Carlos Parra Escobar
% rol: 201973094-0
% ramo: Algoritmos y Complejidad
% semestre: 1ero

\begin{document}

\maketitle
\begin{center}
      \begin{tabular}{|l|r|}
        \hline
        \multicolumn{1}{|c|}{\textbf{Concepto}} &
          \multicolumn{1}{c|}{\textbf{Tiempo [min]}} \\
        \hline
        Revisión & 10\\
        \hline
        Desarrollo    & 7 \\
        \hline
        Informe       & 10 \\
        \hline
      \end{tabular}
    \end{center}
    

 
\section{Instalación}    

Copie el archivo de definición \emph{greedy.cc} en la carpeta de trabajo.

\section{Compilación}

En su línea de comandos favorita tipee el comando \emph{make}.

\section{Uso}

Utilice el programa \emph{tst_greedy}

\end{document}
