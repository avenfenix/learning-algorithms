\documentclass{article}
\usepackage[utf8]{inputenc}

\newcommand{\tnum}{0}
\title{Tarea \tnum\\
      Algoritmos y Complejidad\\[2ex]
      }
\author{
        \textbf{Juan Parra Escobar}\\
        \textbf{201973094-0}
}

\date{17/03/2022}

% nº tarea: 0
% nombre: Juan Carlos Parra Escobar
% rol: 201973094-0
% ramo: Algoritmos y Complejidad
% semestre: 1ero

\begin{document}

\maketitle
\begin{center}
      \begin{tabular}{|l|r|}
        \hline
        \multicolumn{1}{|c|}{\textbf{Concepto}} &
          \multicolumn{1}{c|}{\textbf{Tiempo [min]}} \\
        \hline
        Revisión & 15\\
        \hline
        Desarrollo    & 20 \\
        \hline
        Informe       & 20 \\
        \hline
      \end{tabular}
    \end{center}
    

    
\section{Compilación}

En su línea de comandos favorita tipee el comando \emph{make}.

\section{Uso}

Para usar el programa escriba una lista de numeros con un espacio entre ellos, pulse \emph{Enter} y los numeros aparecerán ordenados.

\end{document}
